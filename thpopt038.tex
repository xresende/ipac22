\documentclass[a4paper,
               %boxit,        % check whether paper is inside correct margins
               %titlepage,    % separate title page
               %refpage       % separate references
               %biblatex,     % biblatex is used
               keeplastbox,   % flushend option: not to un-indent last line in References
               %nospread,     % flushend option: do not fill with whitespace to balance columns
               %hyphens,      % allow \url to hyphenate at "-" (hyphens)
               %xetex,        % use XeLaTeX to process the file
               %luatex,       % use LuaLaTeX to process the file
               ]{jacow}
%
% ONLY FOR \footnote in table/tabular
%
\usepackage{pdfpages,multirow,ragged2e} %
%
% CHANGE SEQUENCE OF GRAPHICS EXTENSION TO BE EMBEDDED
% ----------------------------------------------------
% test for XeTeX where the sequence is by default eps-> pdf, jpg, png, pdf, ...
%    and the JACoW template provides JACpic2v3.eps and JACpic2v3.jpg which
%    might generates errors, therefore PNG and JPG first
%
\makeatletter%
	\ifboolexpr{bool{xetex}}
	 {\renewcommand{\Gin@extensions}{.pdf,%
	                    .png,.jpg,.bmp,.pict,.tif,.psd,.mac,.sga,.tga,.gif,%
	                    .eps,.ps,%
	                    }}{}
\makeatother

% CHECK FOR XeTeX/LuaTeX BEFORE DEFINING AN INPUT ENCODING
% --------------------------------------------------------
%   utf8  is default for XeTeX/LuaTeX
%   utf8  in LaTeX only realises a small portion of codes
%
\ifboolexpr{bool{xetex} or bool{luatex}} % test for XeTeX/LuaTeX
 {}                                      % input encoding is utf8 by default
 {\usepackage[utf8]{inputenc}}           % switch to utf8

\usepackage[USenglish]{babel}
\usepackage{caption}
\usepackage{subcaption}


%
% if BibLaTeX is used
%
\ifboolexpr{bool{jacowbiblatex}}%
 {%
  \addbibresource{jacow-test.bib}
  \addbibresource{biblatex-examples.bib}
 }{}
\listfiles

%%
%%   Lengths for the spaces in the title
%%   \setlength\titleblockstartskip{..}  %before title, default 3pt
%%   \setlength\titleblockmiddleskip{..} %between title + author, default 1em
%%   \setlength\titleblockendskip{..}    %afterauthor, default 1em

\begin{document}

\title{Sirius Injection Optimization}

\author{X.R. Resende\thanks{ximenes.resende@lnls.br}, F.H. de Sá, J.V. Quentino, M.B. Alves, A.C.S. Oliveira, L. Lin}
	
\maketitle

%
\begin{abstract}
   Sirius is the new 3 GeV storage ring (SR)-based 4th generation synchrotron light source built and operated by the Brazilian Synchrotron Light Laboratory (LNLS) located in the CNPEM campus, in Campinas. The foreseeable move to a top-up injection scheme demands improvement of injection efficiency and repeatability levels. In this work we report on the latest efforts in optimizing the Sirius injection system.
\end{abstract}


\section{INTRODUCTION}

The Sirius injector is comprised of a 150 MeV linac with 3-GHz RF structures and 500 Mhz sub-harmonic buncher, a full-energy booster (BO) that shares the tunnel with the storage ring (SR) and low- and high-energy transport lines, LTB and BTS respectively. Booster and SR are served by the same RF master frequency generator.

\begin{table}[!hbt]
   \centering
   \caption{Relevant Sirius Parameters}
   \begin{tabular}{lcc}
      \toprule
      \textbf{Parameter} & \textbf{Value} & \textbf{Units}\\
       \midrule
        RF Freq. & 499.67 & MHz \\ %[3pt]
        BO harm. nr. & 828 & \\
        SR harm. nr. & 864 & \\
        BO rev. time & 1.657 & $\mu$s \\
        SR rev. time & 1.729 & $\mu$s \\
        BO repetition rate & 2 & Hz \\
        BO natural emittance & 3.6 & nm.rad \\
        BO residual coupling & 0.6 & \% \\ 
        EGun multi-bunch charge & 3.0 & nC \\
       \bottomrule
   \end{tabular}
   \label{sirius-params}
\end{table}

The SR is currently operating in decay mode, with fill-ups to 100 mA current values twice a day. Sirius is planned to move to top-up mode before the completion of Sirius Phase-I, scheduled for 2023\cite{liu:IPAC22}. For this new injection scheme to become feasible accelerator teams have been putting considerable effort in the past year on improving the injection efficiencies at various stages, aiming at reduced required injection times and radiation from beam losses, mainly in high energy.

Low-level RF linac parameters optimization, such as sub-harmonic buncher and klystron phases and amplitudes, have been constantly optimized during the past months, as well as Booster injection, ramp and extraction to the SR. In particular, we discuss in the next sections two of the major developments in this direction, both related to the booster: girder realignment and an emittance exchange scheme implemented in the energy ramp. We also describe some of the pending injection issues we are yet to tackle in the next months, before top-up operation. 

\begin{table}[!hbt]
   \centering
   \caption{Beam transmission efficiency for each stage in the injector (1nC/pulse)}
   \begin{tabular}{lc}
      \toprule
      \textbf{Accelerator} & \textbf{Efficiency [\%]} \\
       \midrule
        Linac & $>$ 95 \\ %[3pt]
        LTB & $>$ 95 \\
        Booster & $\sim$ 70 \\
        BTS & $>$ 95 \\
        SR Injection & $\sim$ 95 \\
       \bottomrule
   \end{tabular}
   \label{fig:inj-eff}
\end{table}

The current status of our injection efficiencies after the improvements described in this work is shown in Table\ref{fig:inj-eff}. The overall efficiency of the injector is around 60\% and most of the losses happen in low energies, at the booster injection. For higher EGun charges the efficiency can drop, as discussed below, but the injector still can provide 1 nC/pulse to the SR, a conservative charge value for the envisaged top-up scheme.

\section{BOOSTER REALIGNMENT}

A major booster realignment was performed in the beginning of 2022 to match its circumference to the storage ring RF frequency, thus reducing the off-energy orbit at low energy of the injected beam from the linac in the first turns in the booster. Each booster girder was moved inwards by 158 $\mu$m, corresponding to a reduction of 1 mm in circumference. Before the realignment, the off-energy orbit was, according to BPM-averaged turn-by-turn (TbT) data, estimated to be -0.88 mm. After realignment this average was reduced to -0.34 mm (Fig. \ref{fig:bo-offenergy}). For a 22 cm dispersion function at BPMs, these horizontal position averages translate to -0.40\% and -0.15\% off-energy errors, respectively.
\begin{figure}[!htb]
   \centering
   \includegraphics*[width=.9\columnwidth]{THPOPT038f1.pdf}
   \caption{Effect of booster realignment on the average horizontal position. BPM acquisition at 1 kHz during energy ramp}
   \label{fig:bo-offenergy}
\end{figure}

Apart from reducing the off-energy error, the entire booster ramp was also optimized, mainly by measuring and correcting beam orbit and tunes along the ramp to nominal values. With these improvements the overall booster ramp efficiency is presently optimized to 70\%, from about its 20\% value before (Fig.\ref{fig:bo-ramp})

\begin{figure}[!htb]
   \centering
   \includegraphics*[width=.9\columnwidth]{THPOPT038f2.pdf}
   \caption{Improvement on booster capture efficiency after realignment and ramp optimizations of Jan/2022, using BPM TbT sum signal as proxy to beam current TbT evolution. Injected charge is 0.2 nC, typical in user shift injections.}
   \label{fig:bo-ramp}
\end{figure}


\section{EMITTANCE EXCHANGE}

The other important injection optimization was the transverse emittance exchange (TEE) implemented in the booster ramp. Currently the non-linear dynamics in Sirius is not yet optimized, chromatic sextupole strengths are still set to nominal design values. As a consequence, the measured dynamical aperture corresponds to a horizontal aperture of -8.5 mm, somewhat far from the nominal value of -9.5 mm. For this reason the incoming beam is injected at -8 mm, and not at -9 mm as it should, where the flat-top field of the off-axis non-linear kicker (NLK) is designed. Even for the low emittance Sirius booster, injection away from NLK flat-top position introduces a kick spread that reduces the injection efficiency into the SR. The TEE in the booster right before extraction helps improving injection in the SR by reducing the kick spread from the NLK with the reduced horizontal injected beam size. Implementation details of TEE can be found here\cite{quentino:IPAC22}.

The emittance exchange is accomplished via a transverse-tune crossing implemented in the booster power supplies ramp approximately 1 ms before the extraction instant. Fig.\ref{fig:tee-sigmas} shows measured beam sizes along the emittance exchange process.
\begin{figure}[!htb]
   \centering
   \includegraphics*[width=.9\columnwidth]{THPOPT038f3.png}
   \caption{Fitted beam sizes at one YAG screen of the BTS for different stages in the TEE.}
   \label{fig:tee-sigmas}
\end{figure}

Fig. \ref{fig:tee-screens} shows an example of the CCD beam image recorded from the first YAG screen in the BTS, right after the beam left the booster. It corresponds to two instants in the process: one where the beam was extracted before TEE could have set in (first point in Fig. \ref{fig:tee-sigmas}) and another when the beam is extracted when exchange is optimal (data point at $\sim$0.5 ms in Fig. \ref{fig:tee-sigmas})

% \begin{figure}[!htb]
%   \centering
%   \begin{minipage}[b]{0.9\columnwidth}
%      \includegraphics*[width=.45\columnwidth]{THPOPT038f4a.png}
%      \includegraphics*[width=.45\columnwidth]{THPOPT038f4b.png}
%   \end{minipage}
%   \caption{YAG screen beam images in the entrance of BTS, right after BO extraction. Left: beam extracted before TEE set-in. Right: beam extracted with TEE.}
%   \label{fig:tee-screens}
% \end{figure}

\begin{figure}[!htb]
   \centering
   \begin{subfigure}[b]{0.22\textwidth}
         \centering
         \includegraphics*[width=\columnwidth]{THPOPT038f4a.png}
         \caption{without TEE}
         \label{fig:tee-screens-a}
   \end{subfigure}
   \begin{subfigure}[b]{0.22\textwidth}
         \centering
         \includegraphics*[width=\columnwidth]{THPOPT038f4b.png}
         \caption{with TEE}
         \label{fig:tee-screens-b}
   \end{subfigure}
   \caption{YAG screen beam images in the entrance of BTS, right after BO extraction.}
   \label{fig:tee-screens}
\end{figure}

The TEE implemented in March 2002 has been able to increase injection efficiency on optimal SR conditions by around 10\%. But TEE is still an on-going development: preliminary observations indicate that it also has a positive impact on injection efficiency fluctuations due to jitters and drifts of the pulse electronics and magnets, but its full characterization is lacking.

\section{PENDING ISSUES}
The main pending injector identified issues are connected to the non-repeatability of the booster to storage ring injection conditions, reducing the injection efficiency over time. The subsections below these issues will be described.

\subsection{Septa Temperatures Variations}
When injection starts there is a fast temperature rise up to 60 $^\circ$C within 1-2 minutes for the SR injection septa. During this time would there be a lot of injected beam losses and, for the purpose of minimizing integrated radiation doses, the electron gun (EGun) usually is pulsed moments after the pulsed magnets, giving time for the septa to overcome this fast temperature rise. But after a few minutes of pulsing septa their temperatures enter a new regime where there is a slow but seemingly non-negligible variation over a much longer period. User's shifts injections happen during this regime and SI efficiency changes as septa temperatures reach higher values, as can bee seen in Fig.\ref{fig:septa-temperatures}.
\begin{figure}[!htb]
   \centering
   \includegraphics*[width=.9\columnwidth]{THPOPT038f5.pdf}
   \caption{A full 100 mA injection showing correlation of SR injection efficiency with septa temperature variations. As temperatures increase, reaching the condition optimized in machine studies, the efficiency tends to increase along. The solid orange curve represents beam current.}
   \label{fig:septa-temperatures}
\end{figure}

Usually parameters such as injection beam position and angle (implemented with septa magnets, as well as with static corrector magnets) are weekly optimized during machine studies to accommodate electronics drift and other unknown sources. At machine studies the injection system is turned on for long periods of time in the search of optimized parameters. It is a this hot regime that injection is optimized. But in user's shift the operators can not wait for septa to reach hot temperatures and then, efficiency is compromised. During commissioning in previous years this temperature-dependence had been observed and partially mitigated with improvised air-cooling setup blowing on septa. Further mitigation alternatives are being studied such as temperature to position and angle feed-forward tables and water-cooled septa coils\cite{sergio:IPAC22} 

\subsection{Booster Ramp "Drops" and "Flips"}

Another issue with the booster are systematic drops in its ramp efficiency. Typically they are from about 70\% to about 40\%, with a quasi-periodicity of approximately 6 pulses (3s). This issue was not investigated yet but a natural suspect is synchronization slippages between different power supply (PS) types that can introduce tracking errors. Currently power supplies are synchronized only at the start of each booster cycle, not at every setpoint along the ramp, and within a pulse-width modulation (PWM) period. Since each PS type runs on different PWM clocks the drop quasi-periodicity may be related to a frequency beating of these clocks. Orbit signatures from BPM TbT acquisitions may provide clues to possible power supplies culprits. When one ramp efficiency drop happens it also afects the SR injection, with a considerable reduction of the surviving injected charge. This effect can be observed as deeps in Fig.\ref{fig:septa-temperatures}.

There is yet another issue with the booster that also has an impact on the its ramp quality: sometimes the booster ramp seems to change, flipping from what seems to be two possible states. When this flipping happens it usually stays in that state for days, or sometimes weeks, with a considerable deterioration of the ramp efficiency, if not corrected. A flip in the booster ramp state induces a vertical orbit distortion in low energies with $\sim$1 mm average whose correction restores the ramp efficiency. Two ramp configurations are then needed to be stored and optimized at all times, one for each flip ramp state.

\subsection{Impedance effects in Booster Ramp}

A final issue to be reported here is the influence of the incoming beam charge on the BO ramp efficiency. Beam charge from linac can be controlled from 0 to 3 nC/pulse with the electron gun bias voltage. Daily injection is usually done using bias leading to 0.2-0.5 nC/pulse. But an impedance effect has been observed that for higher charges the ramp efficiency systematically smaller\ref{fig:bo-inj-impedance}.
\begin{figure}[!htb]
   \centering
   \includegraphics*[width=.9\columnwidth]{bo-inj-impedance.pdf}
   \caption{Booster ramp efficiency as a function of injected beam charge. The efficiency is calculated from 2000 TbT BPM sum signal data, averaged over 5 acquisitions for each beam charge.}
   \label{fig:bo-inj-impedance}
\end{figure}

\section{CONCLUSION}

Linac low-level RF adjustments, Booster realignment and orbit and tune optimizations in the energy ramp were essential for increasing the injection efficiency from linac to the SR from below 20\% to above 60\% in 2022. Nowadays the injector system allows for top-up operation as conservatively designed.

Notwithstanding these improvements, many of the issues described above are still pending. They should be tackled in order to provide a cleaner injection from the point of view of radiation doses and to allow for more convenient top-up operation, with longer injection intervals.

%
% only for "biblatex"
%
\ifboolexpr{bool{jacowbiblatex}}%
	{\printbibliography}%
	{%
	% "biblatex" is not used, go the "manual" way
	
	%\begin{thebibliography}{99}   % Use for  10-99  references
	\begin{thebibliography}{9} % Use for 1-9 references
	
	\bibitem{liu:IPAC22}
    L. Lin et al., \textquotedblleft{Status of Sirius Operation}\textquotedblright,
  presented at the 13th International Particle Accelerator Conf. (IPAC’22), Bangkok, Thailand, Jun. 2022, paper TUPOMS002, this conference.
  
    \bibitem{quentino:IPAC22}
    J.V. Quentino et al., \textquotedblleft{Emittance Exchange at Sirius Booster for Storage Ring Injection Improvement}\textquotedblright,
  presented at the 13th International Particle Accelerator Conf. (IPAC’22), Bangkok, Thailand, Jun. 2022, paper THPOPT056, this conference.
  
  \bibitem{sergio:IPAC22}
    S.R. Marques et al., \textquotedblleft{Improvements on Sirius Beam Stability}\textquotedblright,
  presented at the 13th International Particle Accelerator Conf. (IPAC’22), Bangkok, Thailand, Jun. 2022, paper MOPOPT002, this conference.
  
  
  
	\end{thebibliography}

} % end \ifboolexpr
%
% for use as JACoW template the inclusion of the ANNEX parts have been commented out
% to generate the complete documentation please remove the "%" of the next two commands
% 
%\newpage

%\include{annexes-A4}

\end{document}
